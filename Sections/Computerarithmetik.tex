\section{Computerarithmetik}
\subsection{Newton-Raphson}
Sei $g(x)$ die zu Approximierende Funktion. Mit der Definition von $f(y) = g^{-1}(y) - x \eqi 0$ lässt sich nun Schritt für Schritt approximieren:
\[
y_{k+1} = y_k - \frac{f(y_k)}{\frac{\partial}{\partial y}f(y_k)}
\]

\textbf{Beispiel} $g(x) = \frac{1}{\sqrt{x}} =: y$
Damit ergibt sich $g^{-1}(y) = \frac{1}{y^2}$ und 
\begin{align*}
	f(y) &= g^{-1}(y) - x = \frac{1}{y^2} - x \eqi 0 \\
	\frac{\partial}{\partial y}f(y) &= \frac{-2}{y^3}
\end{align*}

Durch einsetzen ergibt sich,
\[
y_{k+1} = y_k - \frac{\frac{1}{y_k^2} - x}{\frac{-2}{y_k^3}} = 0.5y_k(3 - x\cdot y_k^2)
\]
Was nur mit Subtrktion und Mulipikation berechnet werden kann. Durch intelligente umformung kann nun auch die normale Wurzel $\sqrt{x} = \frac{1}{\sqrt{x}}\cdot x$ berechnet werden, ohne eine Division zu berechnen.

\subsection{Quantisierung Float-Int}
Um mit statt mit float mit Integern zu rechnen, kann man diese zuerst durch multiplizieren von $2^n$ von float auf integer umrechnen. Diese Lineare Funktion mapt zB 0..1 auf 0..255. Anschliessend Berechnung mit int durchführen und am Schluss wieder mit $n$ nach rechts shiften um entsprechende neue Skalierung zu erhalten.